
%%% Local Variables:
%%% mode: latex
%%% TeX-master: t

\documentclass[10pt, a4paper, twoside]{amsart}
\usepackage[english]{babel}
\usepackage[T1]{fontenc}
%\usepackage[utf8x]{inputenc}
\usepackage{amsmath, amssymb, amsthm, amsfonts, mathrsfs, amsfonts}
\usepackage{mathtools} 

\usepackage{enumerate}

\usepackage[noabbrev]{cleveref}

%\usepackage{subfig}
\usepackage{pgf,tikz}
%\usetikzlibrary{arrows}

%\usepackage{natbib}
\usepackage[osf]{mathpazo} % Nice text font
\usepackage{euler} % Very nice mathmode font
\usepackage[many]{tcolorbox}    



\theoremstyle{plain}
\newtheorem{lemma}{Lemma}

\def \E {{\mathbb E}}
% bold
\newcommand{\F}{\ensuremath{\mathbb{F}}}
\newcommand{\N}{\ensuremath{\mathbb{N}}}
\newcommand{\Z}{\ensuremath{\mathbb{Z}}}
\newcommand{\Q}{\ensuremath{\mathbb{Q}}}
\newcommand{\R}{\ensuremath{\mathbb{R}}}
\newcommand{\C}{\ensuremath{\mathbb{C}}}
% 
\def \P {{\mathbb P}}
% calligraphic
\newcommand{\A}{\ensuremath{\mathcal{A}}}
\newcommand{\I}{\ensuremath{\mathcal{I}}}

% Delimiters (requires mathtools package)
\DeclarePairedDelimiter\abs{\lvert}{\rvert}
\DeclarePairedDelimiter\brac[]
\DeclarePairedDelimiter\cbrac\{\}
\DeclarePairedDelimiter\paren()
\DeclarePairedDelimiter{\ip}\langle\rangle
\DeclarePairedDelimiter{\nrm}\lVert\rVert
\DeclarePairedDelimiter{\ceil}\lceil\rceil

% Power set
\newcommand{\Ps}{\ensuremath{\mathcal{P}}}

 
% other things ...
\renewcommand{\c}{\ensuremath{\colon}}
\newcommand{\se}{\ensuremath{\subseteq}}
\renewcommand{\d}{\ensuremath{\ d}}
\newcommand{\Ind}{\ensuremath{\mathbb{1}}}
\newcommand{\im}{\ensuremath{\mathbf{i}}}

\newcommand{\argmin}{\operatorname*{argmin}}
\newcommand{\argmax}{\operatorname*{argmax}}
% \renewcommand\qedsymbol{\rule{1ex}{1ex}}

% course specifics
\renewcommand{\P}{\operatorname*{\ensuremath{\mathbf{P}}}} % probability measure
\newcommand{\Ev}{\operatorname*{\ensuremath{\mathbf{E}}}} %expected value
\newcommand{\Fa}{\ensuremath{\mathcal{F}}} %sigma-algebra
\newcommand{\cH}{\ensuremath{\mathcal{H}}}
\newcommand{\cD}{\ensuremath{\mathcal{D}}}
\newcommand{\cX}{\ensuremath{\mathcal{X}}}
\newcommand{\cY}{\ensuremath{\mathcal{Y}}}
\newcommand{\sgn}{\mathrm{sgn}}

\newcommand{\bw}{\ensuremath{\mathbf{w}}} % weight vector
\newcommand{\bl}{\ensuremath{\mathbf{l}}} % loss vector

% Solutions use a modified proof environment
\newenvironment{solution}
               {\let\oldqedsymbol=\qedsymbol
                \renewcommand{\qedsymbol}{$\blacktriangleleft$}
                \begin{proof}[Solution]}
               {\end{proof}
                \renewcommand{\qedsymbol}{\oldqedsymbol}}
                
\newtcolorbox{definitionbox}[1]{%
    tikznode boxed title,
    enhanced,
    arc=0mm,
    interior style={white},
    attach boxed title to top center= {yshift=-\tcboxedtitleheight/2},
    fonttitle=\bfseries,
    colbacktitle=white,coltitle=black,
    boxed title style={size=normal,colframe=white,boxrule=0pt},
    title={#1}}


\newcommand{\TODO}{\textcolor{red}{\textbf{!!!!!! }}}

\begin{document}
\begin{center}
    {\huge\bf Machine learning theory}\\
    {\large\sc Homework set 6 }\\ \vspace{1em}
    {Aras} \textsc{ {Selvi}} \\
    {Daniel} \textsc{ {Figueroa}}\\
    {Fredrik} \textsc{ {Gustafsson}}\\
    {Lasse} \textsc{ {Vuursteen}}\\
    {Martijn} \textsc{ {Bouman}}\\
    {Ruud} \textsc{ {Nimour}}\\
    {Tamar} \textsc{ {Huygen}}\\
    {Twan} \textsc{ {Koperberg}}\\
    {Vincent} \textsc{ {Hennink}}\\ 
    \bigskip
    \today \bigskip
    \hrule
    \bigskip
\end{center}

\section{Notation and Definitions}
\begin{definitionbox}{Definition 1 (Mix-loss Game).}
\textbf{Protocol:} For $t = 1,2, \ldots$
\begin{itemize}
    \item Learner chooses a distribution $\mathbf{w}_t \in \triangle_K$ on K experts.
    \item Adversary reveals loss vector $\mathbf{l}_t \in (-\infty , \infty]^K$.
    \item Learner's loss is the mix loss $-\ln \paren*{\sum_{k=1}^K w_t^k e^{-l_t^k}}$.
\end{itemize}
%End for.\\
\textbf{Objective:} Regret w.r.t. best expert after $T$ rounds:
\begin{equation*}
    R_T = \sum^T_{t=1} - \ln \paren*{\sum_{k=1}^K w_t^k e^{-l_t^k}} - \min_k \sum_{t=1}^T l_t^k .
\end{equation*}
\end{definitionbox}
\begin{definitionbox}{Definition 2 (dot-loss Game).}
\textbf{Protocol:} For $t = 1,2, \ldots$
\begin{itemize}
    \item Learner chooses a distribution $\mathbf{w}_t \in \triangle_K$ on K experts.
    \item Adversary reveals loss vector $\mathbf{l}_t \in [0,1]^K$.
    \item Learner's loss is the dot product $\ip{\mathbf{w}_t , \mathbf{l}_t}$.
\end{itemize}
%End for.\\
\textbf{Objective:} Regret w.r.t. best expert after $T$ rounds:
\begin{equation*}
    R_T = \sum^T_{t=1} \ip{\mathbf{w}_t , \mathbf{l}_t} - \min_k \sum_{t=1}^T l_t^k .
\end{equation*}
\end{definitionbox}

\section{Exercises}
\subsection*{1. Variational (Mirror Descent) form of the AA}
The Aggregating Algorithm plays $w_1^k = \frac{1}{K}$ and updates as
\begin{align*}
    w_{t+1}^k &= \frac{w_t^k e^{-l_t^k}}{\sum_{j=1}^K w_t^j e^{-l_t^j}}. &&\mbox{(AA, incremental)}
\end{align*}
Let us define the \textit{Kullback-leibler divergence} aka \textit{relative entropy} (notion of distance between probability distributions) from $\mathbf{p} \in \triangle_K$ to $\mathbf{q} \in \triangle_K$ by
\begin{equation*}
    \mbox{KL}(\mathbf{p}, \mathbf{q}) = \sum_{k=1}^K p_k \ln \frac{p_k}{q_k}.
\end{equation*}
Fix $\mathbf{w}_t \in \triangle_K$ and $\mathbf{l}_t \in \R^K$. Consider the minimisation problem 
\begin{equation} \label{eq:min}
    \min_{\mathbf{w} \in \triangle_K} \mathbf{w}^T\mathbf{l}_t + \mbox{KL}(\mathbf{w}, \mathbf{w_t})
\end{equation}
\subsubsection*{a}
Show that the minimiser of problem (\ref{eq:min}) is $\mathbf{w_{t+1}}$.
\subsubsection*{b}
Show that the value of problem (\ref{eq:min}) is the mix loss.
\begin{solution} 
 By part (\textit{a}), we have that
 \begin{align*}
  \min_{\bw \in \triangle_{K}} \ip{\bw,\bl_t} + \mathrm{KL}(\bw,\bw_t) 
  &= \ip{\bw_{t+1},\bl_t} + \mathrm{KL}(\bw_{t+1},\bw_t) \\
  &= \ip{\bw_{t+1},\bl_t} + \sum_{k=1}^K w_{t+1}^{k} \log\paren*{\frac{w_{t+1}^{k}}{w_{t}^{k}}} \\
  &= \sum_{k=1}^K w_{t+1}^{k} \paren*{\log\paren*{\frac{w_{t+1}^{k}}{w_{t}^{k}}} + l_t^k} \\
  &= \sum_{k=1}^K w_{t+1}^{k} \paren*{\log\paren*{\frac{e^{-l_t^k}}{\sum_{j=1}^K w_t^j e^{-l_t^j}}} + l_t^k} \\
  &= -\log\paren*{\sum_{j=1}^K w_t^j e^{-l_t^j}}\sum_{k=1}^K w_{t+1}^{k}  \\
  &= -\log\paren*{\sum_{j=1}^K w_t^j e^{-l_t^j}},
 \end{align*}
which is the mix-loss
\end{solution}


\subsubsection*{Big picture:}
We can see another motivation for the AA. It chooses te weights $\mathbf{w}_{t+1}$ to balance the fit on the new losses $\mathbf{l}_t$ with proximity to the previous weights $w_t$, which encode all information learned from the past. 

\subsection*{2. Adapting to $T$ for dot loss.}
We saw in the lecture that the Hedge algorithm (for the dot-loss game) with learning rate $\eta = \sqrt{\frac{8\ln K}{T}}$ has regret after $T$ rounds bounded by $\sqrt{T/ 2 \ln K}$. In practice, we may not know $T$ in advance, or we may even desire an algorithm that has good guarantees for all $T$ simultaiously, i.e. that keeps on operating forever.

Consider the following exponential (base $3$) restarting schedule to accomplish this. We run Hedge for $1$ round, with $\eta$ tuned for $1$ round. After that, we restart Hedge for $3$ rounds with $\eta$ tuned for $3$ rounds. After that, we restart Hedge again after $9$ rounds with $\eta$ tuned for $9$ rounds, and so on.

Prove that the overall accumulated regret of Hedge with this scheme is bounded above by a universal constant times $\sqrt{T \ln K}$. (Your argument should work for $T$ that are not a power of $3$).

\begin{solution}
 Let $R^{H}_T$ denote the regret of the Hedge algoritm at time $T$.
 Let $R^{A}_T$ denote the total regret incurred by the above algorithm until time $T$.
 We will apply induction to $T$. For each $T \in \N$ we have that $R_T \leq \sqrt{\tfrac{t}{2}\log(K)}$.
 So, for $T=1$ we have that 
 \begin{align*}
  R^A_1 &= R^H_1 \\
  & \leq \sqrt{\tfrac{1}{2}\log(K)} \\
  & \leq \sqrt{\log(K)}.
  \end{align*}
Thus for $T=1$ the statement holds.

Let $T > 1$ be given and assume the statement holds for all smaller $T$.
 Let $H_{n,m}$ denote the $n$-th generalized harmonic number of order $m$, i.e.
 \begin{equation*}
  H_{n,m}=\sum_{k=1}^{n}\frac{1}{k^m}.
 \end{equation*}
 Then we have that
 \begin{align*}
  R^{A}_{T} &\leq \sum_{n=0}^{\ceil{\log_3(2T + 1)} - 1} \sum_{t=1}^{3^n} \sqrt{\tfrac{t}{2}\log(K)} \\
  &=\sqrt{\frac{\log(K)}{2}} \sum_{n=0}^{\ceil{\log_3(2T + 1)} - 1} \sum_{t=1}^{3^n} \sqrt{t} \\
  &=\sqrt{\frac{\log(K)}{2}} \sum_{n=0}^{\ceil{\log_3(2T + 1)} - 1} H_{3^n,2} \\
 \end{align*}
 \TODO
\end{solution}


\subsection*{3. Dot loss lower bound.}
Consider the $K = 2$ expert version of the $T$-round dot loss game (Definition ). In this exercise we will prove that the worst-case expected regret is at least of order $\sqrt{T}$. Consider an adversary that for each $t = 1, \ldots, T$ assigns loss vector $\mathbf{l}_t = (0,1)$ or $\mathbf{l}_t = (1,0)$ i.i.d. uniformly at random.
\subsubsection*{a}
\begin{enumerate}[i]
    \item Show that the expected loss of any learner is $T/2$.
    \item Show that $2(1/2 - l^k_t)$ is Rademacher for each $k \in \cbrac{1,2}$.
    \item Show that $\sum_{t=1}^T (1/2 - l_t^2) = - \sum_{t=1}^T(1/2 - l_t^1)$.
\end{enumerate}
\subsubsection*{b} Argue that the expected loss of the best expert is bounded above by \\ 
$\Ev \left[ \min_k \sum_{t=1}^T l_t^k \right] \leq T/2 - c \sqrt{T}$ for some $c>0$. You can use the following fact: Let $X_1, \ldots , X_T$ be i.i.d. Rademacher random variables. Then
\begin{equation*}
    \Ev \abs*{\sum_{t=1}^T X_t} \in \left[ \sqrt{\frac{2(T-1)}{\pi}}, \sqrt{\frac{2(T+1)}{\pi}} \right].
\end{equation*}
\end{document}

%%% End:
