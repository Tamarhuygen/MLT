\documentclass[10pt, a4paper, twoside]{amsart}
\usepackage[english]{babel}
\usepackage[T1]{fontenc}
%\usepackage[utf8x]{inputenc}
\usepackage{amsmath, amssymb, amsthm, amsfonts, mathrsfs, amsfonts}
\usepackage{mathtools} 

\usepackage{enumerate}

\usepackage[noabbrev]{cleveref}

%\usepackage{subfig}
\usepackage{pgf,tikz}
%\usetikzlibrary{arrows}

%\usepackage{natbib}
\usepackage[osf]{mathpazo} % Nice text font
\usepackage{euler} % Very nice mathmode font


% bold
\newcommand{\F}{\ensuremath{\mathbb{F}}}
\newcommand{\N}{\ensuremath{\mathbb{N}}}
\newcommand{\Z}{\ensuremath{\mathbb{Z}}}
\newcommand{\Q}{\ensuremath{\mathbb{Q}}}
\newcommand{\R}{\ensuremath{\mathbb{R}}}
\newcommand{\C}{\ensuremath{\mathbb{C}}}
% 
% calligraphic
\newcommand{\A}{\ensuremath{\mathcal{A}}}
\newcommand{\E}{\ensuremath{\mathcal{E}}}
\newcommand{\I}{\ensuremath{\mathcal{I}}}

% Delimiters (requires mathtools package)
\DeclarePairedDelimiter\abs{\lvert}{\rvert}
\DeclarePairedDelimiter\brac[]
\DeclarePairedDelimiter\cbrac\{\}
\DeclarePairedDelimiter\paren()
\DeclarePairedDelimiter{\ip}\langle\rangle
\DeclarePairedDelimiter{\nrm}\lVert\rVert
\DeclarePairedDelimiter{\ceil}\lceil\rceil

% Power set
\newcommand{\Ps}{\ensuremath{\mathcal{P}}}

 
% other things ...
\renewcommand{\c}{\ensuremath{\colon}}
\newcommand{\se}{\ensuremath{\subseteq}}
\renewcommand{\d}{\ensuremath{\ d}}
\newcommand{\Ind}{\ensuremath{\mathbb{1}}}
\newcommand{\im}{\ensuremath{\mathbf{i}}}

\newcommand{\argmin}{\operatorname*{argmin}}
\newcommand{\argmax}{\operatorname*{argmax}}
% \renewcommand\qedsymbol{\rule{1ex}{1ex}}

% course specifics
\renewcommand{\P}{\operatorname*{\ensuremath{\mathbf{P}}}} % probability measure
\newcommand{\Ev}{\operatorname*{\ensuremath{\mathbf{E}}}} %expected value
\newcommand{\Fa}{\ensuremath{\mathcal{F}}} %sigma-algebra
\newcommand{\cH}{\ensuremath{\mathcal{H}}}
\newcommand{\cD}{\ensuremath{\mathcal{D}}}
\newcommand{\cX}{\ensuremath{\mathcal{X}}}
\newcommand{\sgn}{\mathrm{sgn}}

% Solutions use a modified proof environment
\newenvironment{solution}
               {\let\oldqedsymbol=\qedsymbol
                \renewcommand{\qedsymbol}{$\blacktriangleleft$}
                \begin{proof}[Solution]}
               {\end{proof}
                \renewcommand{\qedsymbol}{\oldqedsymbol}}


\newcommand{\TODO}{\textcolor{red}{\textbf{!!!!!! }}}

\newcommand{\firstName}  {Twan}
\newcommand{\lastName}   {Koperberg}
\newcommand{\studId}     {0713309 (Leiden)}
\renewcommand{\email}    {twankop@gmail.com}

\newcommand{\firstNameII}  {Tamar}
\newcommand{\lastNameII}   {Huygen}
\newcommand{\studIdII}     {10907483 (UvA)}
\newcommand{\studIdIII}    {2556474 (VU)}
\newcommand{\emailII}     {tamar@huygen.nl}

\begin{document}
\begin{center}

  {\huge\bf Machine learning theory}\\
  {\large\sc Homeworkset 3 }\\ \vspace{1em}
  \firstName \textsc{ \lastName}, {\sc s}\studId \\
  \email\text{}\\ \smallskip
  \firstNameII \textsc{ \lastNameII}, \studIdII, \studIdIII\\
  \emailII \\ \bigskip
  \today \\\bigskip
  \hrule
  \bigskip
 \end{center}

\section*{Exercise 6.1}
\subsection*{1}
Show that following monotonicity property of VC-dimension: For every two hypothesis classes if $\cH' \subseteq \cH$ then $\text{VCdim}\paren*{\cH'}\leq \text{VCdim}\paren*{\cH}$.
\begin{solution}
  Suppose $\text{VCdim}(\cH') < \infty$, then by definition of $\text{VCdim}$ there exists a subset $C \subseteq \cX$ with a maximal size that is still shatered by $\cH'$ and $\text{VCdim}(\cH') = \abs{C}$. By the definition of shattering, we know that the restriction of $\cH'$ on $C$ is the set of all functions from $C$ to $\cbrac{0,1}$. Because $\cH' \subseteq \cH$, we know that all these functions are also in the restriction from $\cH$ on $C$. This means that $C$ is also shattered by $\cH$. So now we have that the maximal size of the subset of $\cX$ that is shattered by $\cH$ is greater or equal than $\abs{C}$. So we have that $\text{VCdim}(\cH') = \abs{C} \leq \text{VCdim}(\cH)$. \\
  In the case that $\text{VCdim}(\cH') = \infty$, $\cH'$ can shatter sets of arbitrary large size. And because $\cH' \subseteq \cH$, $\cH$ can also shatter sets of arbitrary large size, so then  $\text{VCdim}(\cH ) = \infty$
 \TODO
\end{solution}
\section*{Exercise 6.2}
Given some finite domain set, $\cX$, and a number $k \leq \abs{\cX}$, figure out the VC-dimension of each of the following classes (and prove your claims):
\subsection*{1}
$\cH^{\cX}_{=k} = \cbrac{h \in \cbrac{0,1}^{\cX}: \abs{\cbrac{x: h(x) = 1}}=k}$. That is, the set of all functions that assign the value $1$ to exactly $k$ elements of $\cX$.
\begin{solution}
\TODO
\end{solution}
\subsection*{2}
$\cH_{at-most-k} = \cbrac{h \in \cbrac{0,1}^{\cX}: \abs{\cbrac{x: h(x) = 1}} \leq k} \text{ or }  \abs{\cbrac{x: h(x) = 0}}\leq k$.
\begin{solution}
\TODO
\end{solution}
\section*{Exercise 6.3}
Let $\cX$ be a Boolean hypercube $\cbrac{0,1}^n$. For a set $I \subseteq \cbrac{1,2,\ldots ,n}$ we define a \textit{parity function $h_I$} as follows. On a binary vector $\mathbf{x} = (x_1,x_2,\ldots ,x_n) \in \cbrac{0,1}^n$,
\begin{equation*}
  h_I(\mathbf{x}) = \paren*{\sum_{i \in I}x_i} \mod 2.
\end{equation*}
(That is, $h_I$ computes the parity of bits in $I$.) What is the VC-dimension of the class of all such parity functions, $\cH_{n\text{-parity}} = \cbrac{h_I:I \subseteq \cbrac{1,2, \dots ,n}}$?
\begin{solution}
\TODO
\end{solution}
\section*{Exercise 6.6}
From question 3.4: \\
In this question we study the hypothesis class of \textit{Boolean conjunctions} defined as follows. The instance space is $\cX = \cbrac{0,1}^d$ and the label set is $\mathcal{Y}=\cbrac{0,1}$. A literal over the variables $x_1, \ldots ,x_d$ is a simple Boolean function that takes the form $\mathbf{f}(\mathbf{x}) = x_i$, for some $i \in [d]$, or $\mathbf{f}(\mathbf{x}) = 1-x_i$ for some $i \in [d]$. We use the notation $\bar{x}_i$ as shorthand for $1-x_i$. A conjunction is any product of literals. In Boolean logic, the product is denoted using a $\land$ sign. For example, the function $h(\mathbf{x}) = x_1 \cdot (1-x_2)$ is written as $x_1 \land x_2$.\\
\textbf{VC-dimension of Boolean conjunctions:} Let $\cH^d_{con}$ be the class of boolean conjunctions over the variables $x_1, \ldots , x_d$ $(d \geq 2)$. We already know that this class is finite and thus (agnostic) PAC learnable. In this question we calculate $\text{VCdim}(\cH^d_{con})$.
\subsection*{1}
Show that $\abs{\cH^d_{con}} \leq 3^d + 1$.
\begin{solution}
\TODO
\end{solution}
\subsection*{2}
Conclude that $\text{VCdim}(\cH) \leq d \log 3$.
\begin{solution}
\TODO
\end{solution}
\subsection*{3}
Show that $\cH^d_{con}$ shatters the set of unit vectors $\cbrac{\mathbf{e}_i:i \leq d}$.
\begin{solution}
\TODO
\end{solution}
\subsection*{4}
Show that $\text{VCdim}(\cH_{con}^d) \leq d$.\\
Hint: Assume by contradiction that there exists a set $C = \cbrac{c_1, \ldots , c_{d+1}}$ that is shattered by $\cH^d_{con}$. Let $h_1, \ldots h_{d+1}$ be hypotheses in $\cH^d_{con}$ that satisfy
\begin{equation*}
  \forall i,j \in [d+1], h_i(c_j) = \begin{cases} 0 & i=j \\
    1 & otherwise\end{cases}.
\end{equation*}
For each $i \in [d+1]$, $h_i$ (or more accurately, the conjunction that corresponds to $h_i$) contains some literal $l_i$ which is false on $c_i$ and true on some $c_j$ for each $j \neq i$. Use the Pigeonhole principle to show that there must be a pair $i < j \leq d+1$ such that $l_i$ and $l_j$ use the same $x_k$ and use that fact to derive a contradiction to the requirements from the conjunctions $h_i$, $h_j$.
\begin{solution}
\TODO
\end{solution}
\subsection*{5}
Consider the class $\cH^d_{mcon}$ of monotone Boolean conjunctions over $\cbrac{0,1}^d$. Monotonicity here means that the conjunctions do not contain negations. As in $\cH^d_{con}$, the empty conjunction is interpreted as the all-positive hypothesis. We augment $\cH^d_{mcon}$ with the all-negative hypothesis $h^-$.\\
Show that $\text{VCdim}(\cH_{mcon}^d) = d$.
\begin{solution}
\TODO
\end{solution}

\end{document}




%%% Local Variables:
%%% mode: latex
%%% TeX-master: t
%%% End:
