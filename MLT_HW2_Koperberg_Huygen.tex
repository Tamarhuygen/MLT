\documentclass[10pt, a4paper, twoside]{amsart}
\usepackage[english]{babel}
\usepackage[T1]{fontenc}
%\usepackage[utf8x]{inputenc}
\usepackage{amsmath, amssymb, amsthm, amsfonts, mathrsfs, amsfonts}
\usepackage{mathtools} 

\usepackage{enumerate}

\usepackage[noabbrev]{cleveref}

%\usepackage{subfig}
\usepackage{pgf,tikz}
%\usetikzlibrary{arrows}

%\usepackage{natbib}
\usepackage[osf]{mathpazo}
\usepackage{euler}


% bold
\newcommand{\F}{\ensuremath{\mathbb{F}}}
\newcommand{\N}{\ensuremath{\mathbb{N}}}
\newcommand{\Z}{\ensuremath{\mathbb{Z}}}
\newcommand{\Q}{\ensuremath{\mathbb{Q}}}
\newcommand{\R}{\ensuremath{\mathbb{R}}}
\newcommand{\C}{\ensuremath{\mathbb{C}}}
% 
% calligraphic
\newcommand{\A}{\ensuremath{\mathcal{A}}}
\newcommand{\E}{\ensuremath{\mathcal{E}}}
\newcommand{\I}{\ensuremath{\mathcal{I}}}

% Delimiters (requires mathtools package)
\DeclarePairedDelimiter\abs{\lvert}{\rvert}
\DeclarePairedDelimiter\brac[]
\DeclarePairedDelimiter\cbrac\{\}
\DeclarePairedDelimiter\paren()
\DeclarePairedDelimiter{\ip}\langle\rangle
\DeclarePairedDelimiter{\nrm}\lVert\rVert

% Power set
\newcommand{\Ps}{\ensuremath{\mathcal{P}}}

 
% other things ...
\renewcommand{\c}{\ensuremath{\colon}}
\newcommand{\se}{\ensuremath{\subseteq}}
\renewcommand{\d}{\ensuremath{\ d}}
\newcommand{\Ind}{\ensuremath{\mathbb{1}}}
\newcommand{\im}{\ensuremath{\mathbf{i}}}

% \renewcommand\qedsymbol{\rule{1ex}{1ex}}

% course specifics
\newcommand{\Neighbour}{\ensuremath{\mathcal{N}}}
\renewcommand{\P}{\ensuremath{\mathbf{P}}}
\newcommand{\Ev}{\ensuremath{\mathbf{E}}}
\newcommand{\StochDom}{\ensuremath{\stackrel{\mathcal{D}}{\preceq}}}
\newcommand{\sgn}{\mathrm{sgn}}

% Solutions use a modified proof environment
\newenvironment{solution}
               {\let\oldqedsymbol=\qedsymbol
                \renewcommand{\qedsymbol}{$\blacktriangleleft$}
                \begin{proof}[Solution]}
               {\end{proof}
                \renewcommand{\qedsymbol}{\oldqedsymbol}}


\newcommand{\TODO}{\textcolor{red}{\textbf{!!!!!! }}}

\newcommand{\firstName}  {Twan}
\newcommand{\lastName}   {Koperberg}
\newcommand{\studId}     {0713309 (Leiden)}
\renewcommand{\email}    {twankop@gmail.com}

\newcommand{\firstNameII}  {Tamar}
\newcommand{\lastNameII}   {Huygen}
\newcommand{\studIdII}     {10907483 (UvA)}
\newcommand{\studIdIII}    {2556474 (VU)}
\newcommand{\emailII}     {tamar@huygen.nl}

\begin{document}
\begin{center}
  {\huge\bf Measure learning theory}\\
  {\large\sc Homeworkset 2 }\\ \vspace{1em}
  \firstName \textsc{ \lastName}, {\sc s}\studId \\
  \email\text{}\\ \smallskip
  \firstNameII \textsc{ \lastNameII}, \studIdII, \studIdIII\\
  \emailII \\ \bigskip
  \today \\\bigskip
  \hrule
  \bigskip
\end{center}

\section*{Exercise 2.1}
We have shown that the predictor defined in Equation (2.3) leads to overfitting. 
While this predictor seems to be very unnatural, the goal of this exercise is to show 
that it can be described as a thresholded polynomial. 
That is, show that given a training set 
\begin{equation*}
S = \cbrac{(x_i, f(x_i))}_{i=1}^{m} \se (\R^d \times \cbrac{0,1})^m, 
\end{equation*}
there exists a polynomial $p_S$ such that $h_S(x) = 1$ if and only if $p_S(x)\geq 0$, 
where $h_S$ is as defined in Equation (2.3). It follows that learning the class of all 
thresholded polynomials using the ERM rule may lead to overfitting.
\begin{solution}
Let $h_S:\R^d\to\cbrac{0,1}$ be given by
\begin{equation*}
 h_S(x)=
 \begin{cases}
  f(x_i) & \text{ if there exists }i \in [m] \text{ such that } x=x_i, \\
  0 & \text{ otherwise.} \\  
 \end{cases}
\end{equation*}

Consider the polynomial $p_S:\R^d \to \R$ given by
\begin{equation*}
 p_S(x^1,x^2,\ldots,x^d)=-\prod_{i=1}^{m}\paren*{(1-f(x_i))+\sum_{j=1}^d(x^j-x_i^j)^2},
\end{equation*}
where $x^j$ denotes the $j$-th component of $x \in \R^d$.

Note that $(1-f(x_i))+\sum_{j=1}^d(x^j-x_i^j)^2 \geq 0$ for all $i \in [m]$.
This means that $p_{S}(x)\leq 0$ for all $x \in \R^d$. Also we have that $p_S(x)=0$ if and only if there exists an 
$x_i \in S|_x$ such that $x=x_i$ and $f(x_i)=1$. So, we have that $p_S(x)=0$ if and only if $h_S(x)=1$.
\end{solution}

\section*{Exercise 2.2}
\TODO

\end{document}


