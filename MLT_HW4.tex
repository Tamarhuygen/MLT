\documentclass[10pt, a4paper, twoside]{amsart}
\usepackage[english]{babel}
\usepackage[T1]{fontenc}
%\usepackage[utf8x]{inputenc}
\usepackage{amsmath, amssymb, amsthm, amsfonts, mathrsfs, amsfonts}
\usepackage{mathtools} 

\usepackage{enumerate}

\usepackage[noabbrev]{cleveref}

%\usepackage{subfig}
\usepackage{pgf,tikz}
%\usetikzlibrary{arrows}

%\usepackage{natbib}
\usepackage[osf]{mathpazo} % Nice text font
\usepackage{euler} % Very nice mathmode font


% bold
\newcommand{\F}{\ensuremath{\mathbb{F}}}
\newcommand{\N}{\ensuremath{\mathbb{N}}}
\newcommand{\Z}{\ensuremath{\mathbb{Z}}}
\newcommand{\Q}{\ensuremath{\mathbb{Q}}}
\newcommand{\R}{\ensuremath{\mathbb{R}}}
\newcommand{\C}{\ensuremath{\mathbb{C}}}
% 
% calligraphic
\newcommand{\A}{\ensuremath{\mathcal{A}}}
\newcommand{\E}{\ensuremath{\mathcal{E}}}
\newcommand{\I}{\ensuremath{\mathcal{I}}}

% Delimiters (requires mathtools package)
\DeclarePairedDelimiter\abs{\lvert}{\rvert}
\DeclarePairedDelimiter\brac[]
\DeclarePairedDelimiter\cbrac\{\}
\DeclarePairedDelimiter\paren()
\DeclarePairedDelimiter{\ip}\langle\rangle
\DeclarePairedDelimiter{\nrm}\lVert\rVert
\DeclarePairedDelimiter{\ceil}\lceil\rceil

% Power set
\newcommand{\Ps}{\ensuremath{\mathcal{P}}}

 
% other things ...
\renewcommand{\c}{\ensuremath{\colon}}
\newcommand{\se}{\ensuremath{\subseteq}}
\renewcommand{\d}{\ensuremath{\ d}}
\newcommand{\Ind}{\ensuremath{\mathbb{1}}}
\newcommand{\im}{\ensuremath{\mathbf{i}}}

\newcommand{\argmin}{\operatorname*{argmin}}
\newcommand{\argmax}{\operatorname*{argmax}}
% \renewcommand\qedsymbol{\rule{1ex}{1ex}}

% course specifics
\renewcommand{\P}{\operatorname*{\ensuremath{\mathbf{P}}}} % probability measure
\newcommand{\Ev}{\operatorname*{\ensuremath{\mathbf{E}}}} %expected value
\newcommand{\Fa}{\ensuremath{\mathcal{F}}} %sigma-algebra
\newcommand{\cH}{\ensuremath{\mathcal{H}}}
\newcommand{\cD}{\ensuremath{\mathcal{D}}}
\newcommand{\cX}{\ensuremath{\mathcal{X}}}
\newcommand{\sgn}{\mathrm{sgn}}

% Solutions use a modified proof environment
\newenvironment{solution}
               {\let\oldqedsymbol=\qedsymbol
                \renewcommand{\qedsymbol}{$\blacktriangleleft$}
                \begin{proof}[Solution]}
               {\end{proof}
                \renewcommand{\qedsymbol}{\oldqedsymbol}}


\newcommand{\TODO}{\textcolor{red}{\textbf{!!!!!! }}}

\newcommand{\firstName}  {Twan}
\newcommand{\lastName}   {Koperberg}
\newcommand{\studId}     {0713309 (Leiden)}
\renewcommand{\email}    {twankop@gmail.com}

\newcommand{\firstNameII}  {Tamar}
\newcommand{\lastNameII}   {Huygen}
\newcommand{\studIdII}     {10907483 (UvA)}
\newcommand{\studIdIII}    {2556474 (VU)}
\newcommand{\emailII}     {tamar@huygen.nl}

\begin{document}
\begin{center}

  {\huge\bf Machine learning theory}\\
  {\large\sc Homeworkset 4 }\\ \vspace{1em}
  \firstName \textsc{ \lastName}, {\sc s}\studId \\
  \email\text{}\\ \smallskip
  \firstNameII \textsc{ \lastNameII}, \studIdII, \studIdIII\\
  \emailII \\ \bigskip
  \today \\\bigskip
  \hrule
  \bigskip
 \end{center}

 
 \section*{Exercise 6.7}
 \subsection*{(1)}
 \begin{solution}
 Take $\cH_{=1}^{[0,1]}$ from exercise 6.2. That is, $\cH_{=1}^{[0,1]}$ denotes the collection of functions 
 $h:[0,1]\to \cbrac{0,1}$ such that $\abs{\cbrac{x \in [0,1] \c h(x)=1}}=1$.
 \end{solution}

 \subsection*{(2)}
 \begin{solution}
 Let $\cH$ denote any hypotheses class with $\abs{\cH}=1$. Then $\mathrm{VCdim}(\cH)=0$ and $\log_2(\abs{\cH})=0$.
 \end{solution}
 
 \section*{Exercise 6.8}
 
 \section*{Exercise 6.9}
 \begin{solution}
  We will show that $\mathrm{VCdim}(\cH)=3$. Consider the set $C=\cbrac{1,2,3}$. Let $f:C \to \cbrac{-1,1}$ be a function.
  Take $s = f(2)$, $a=\min{x \in C \c f(x)=f(2)}$ and $b=\max{x \in C \c f(x)=f(2)}$. Then we have that $h_{a,b,s}|_{C}=f$,
  so $C$ is shattered by $\cH$. Let $D=\cbrac{x_1,x_2,x_3,x_4}$ be given. Without loss of generality, we can assume that $x_1<x_2<x_3<x_4$. Define $g:D\to\cbrac{-1,1}$ by
  \begin{equation*}
   g(x)=
   \begin{cases}
    1 & \text{ if }x\in \cbrac{x_1,x_3}, \\
    -1 & \text{ if }x\in \cbrac{x_2,x_4}.
   \end{cases}
  \end{equation*}
Suppose ad absurdum that there exists an $h_{a,b,s} \in \cH$ with $h_{a,b,s}|_{D}=g$. Then if $x_2 \in [a,b]$ it follows that $x_3  \notin [a,b]$ and $x_4 \in [a,b]$, which gives a contradiction. However, we also have that if $x_2 \notin [a,b]$
it follows that $x_1  \in [a,b]$ and $x_3 \in [a,b]$, which gives a contradiction. 
Both cases give a contradiction, hence no such $h_{a,b,s}$ exists.

It follows that no subset of $\R$ with four elements is shattered by $\cH$.  
 \end{solution}

 
 \section*{Exercise 6.11}
 
 
\end{document}




%%% Local Variables:
%%% mode: latex
%%% TeX-master: t
%%% End:
