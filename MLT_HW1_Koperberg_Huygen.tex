\documentclass[10pt, a4paper, twoside]{amsart}
\usepackage[english]{babel}
\usepackage[T1]{fontenc}
%\usepackage[utf8x]{inputenc}
\usepackage{amsmath, amssymb, amsthm, amsfonts, mathrsfs, amsfonts}
\usepackage{mathtools} 
\usepackage{natbib}
\usepackage{enumerate}

\usepackage[noabbrev]{cleveref}

%\usepackage{subfig}
\usepackage{pgf,tikz}
%\usetikzlibrary{arrows}

%\usepackage{natbib}
\usepackage[osf]{mathpazo}
\usepackage{euler}


% bold
\newcommand{\F}{\ensuremath{\mathbb{F}}}
\newcommand{\N}{\ensuremath{\mathbb{N}}}
\newcommand{\Z}{\ensuremath{\mathbb{Z}}}
\newcommand{\Q}{\ensuremath{\mathbb{Q}}}
\newcommand{\R}{\ensuremath{\mathbb{R}}}
\newcommand{\C}{\ensuremath{\mathbb{C}}}
% 
% calligraphic
\newcommand{\A}{\ensuremath{\mathcal{A}}}
\newcommand{\E}{\ensuremath{\mathcal{E}}}
\newcommand{\I}{\ensuremath{\mathcal{I}}}

% Delimiters (requires mathtools package)
\DeclarePairedDelimiter\abs{\lvert}{\rvert}
\DeclarePairedDelimiter\brac[]
\DeclarePairedDelimiter\cbrac\{\}
\DeclarePairedDelimiter\paren()
\DeclarePairedDelimiter{\ip}\langle\rangle
\DeclarePairedDelimiter{\nrm}\lVert\rVert

% Power set
\newcommand{\Ps}{\ensuremath{\mathcal{P}}}

 
% other things ...
\renewcommand{\c}{\ensuremath{\colon}}
\newcommand{\se}{\ensuremath{\subseteq}}
\renewcommand{\d}{\ensuremath{\ d}}
\newcommand{\Ind}{\ensuremath{\mathbb{1}}}
\newcommand{\im}{\ensuremath{\mathbf{i}}}

% \renewcommand\qedsymbol{\rule{1ex}{1ex}}

% course specifics
\newcommand{\Neighbour}{\ensuremath{\mathcal{N}}}
\renewcommand{\P}{\ensuremath{\mathbf{P}}}
\newcommand{\Ev}{\ensuremath{\mathbf{E}}}
\newcommand{\StochDom}{\ensuremath{\stackrel{\mathcal{D}}{\preceq}}}
\newcommand{\sgn}{\mathrm{sgn}}

% Solutions use a modified proof environment
\newenvironment{solution}
               {\let\oldqedsymbol=\qedsymbol
                \renewcommand{\qedsymbol}{$\blacktriangleleft$}
                \begin{proof}[Solution]}
               {\end{proof}
                \renewcommand{\qedsymbol}{\oldqedsymbol}}


\newcommand{\TODO}{\textcolor{red}{\textbf{!!!!!! }}}

\newcommand{\firstName}  {Twan}
\newcommand{\lastName}   {Koperberg}
\newcommand{\studId}     {0713309}
\renewcommand{\email}    {twankop@gmail.com}

\begin{document}
\begin{center}
  {\huge\bf Measure learning theory}\\
  {\large\sc Homework 1 }\\ \vspace{1em}
  \firstName \textsc{ \lastName}, {\sc s}\studId (Leiden)\\
  \email\text{}\\ \smallskip
  Tamar \textsc{ Huygen}, UvA: 109097483, VU: 2556474\\
  tamar@huygen.nl \\ \bigskip
  \today \\\bigskip
  \hrule
  \bigskip
\end{center}

\section*{Exercise 1}
In this question we are following the common special-case notation for $2$ outcomes, 
where outcomes are $\cbrac{0, 1}$ and distributions on these $2$ outcomes are parametrised by the probability 
$q \in [0, 1]$ of observing the outcome $1$.

Determine (prove or disprove) whether the following losses are proper.
\subsection*{(a)}
Take $\ell(0,q)=q$ and $\ell(1,q)=1-q$.
\begin{solution}
The associated risk function is given by
\begin{equation*}
L(p,q)=(1-p)q+p(1-q).
\end{equation*}
Take $p=\tfrac{1}{4}$ and $q=0$. Then we have that $L(p,p)=\tfrac{3}{8}$, but that $L(p,q)=\tfrac{1}{4}$.
Therefore, $\ell$ is not a proper loss function.
\end{solution}

\subsection*{(b)}
Take $\ell(0,q)=q^2$ and $\ell(1,q)=(1-q)^2$.
\begin{solution}
The associated risk function is given by
\begin{equation*}
L(p,q)=(1-p)q^2+p(1-q)^2.
\end{equation*}
Fix $p$ and consider $L_p(q)=L(p,q)$ as a function of $q$.
The derivative of $L_p$ is given by
\begin{equation*}
L_p'(q)=2q-2p.
\end{equation*}
Setting this derivative equal to $0$ yields $q=p$, so $L_p$ has an extremum at the point $q=p$.
The second derivative of $L_p$ is given by $L''(q)=2$.
As this is positive, the extremum at $q=p$ is a minimum. 
As this holds for all $p \in [0,1]$, this shows that $L(p,p) \leq L(p,q)$ for all $p,q \in [0,1]$.
Therefore, $\ell$ is a proper loss function.
\end{solution}

\subsection*{(c)}
Take $\ell(0,q)=\sqrt{\frac{q}{1-q}}$ and $\ell(1,q)=\sqrt{\frac{1-q}{q}}$.
\begin{solution}
The associated risk function is given by
\begin{equation*}
L(p,q)=(1-p)\sqrt{\frac{q}{1-q}}+p\sqrt{\frac{1-q}{q}}.
\end{equation*}
Fix $p$ and consider $L_p(q)=L(p,q)$ as a function of $q$.
The derivative of $L_p$ is given by
\begin{equation*}
L_p'(q)=\frac{q-p}{2(q-q^2)^{\tfrac{3}{2}}}. \TODO
\end{equation*}
We see that $L_p'(p)=0$, so $L_p$ has an extremum at the point $q=p$. \TODO
\end{solution}

\subsubsection*{tamars alternative solution}
We can also look at the following function:
\begin{align*}
  d(p,q) & = L(p,q)-L(p,p)\\
         & = (1-p)\sqrt{\frac{q}{1-q}}+p\sqrt{\frac{1-q}{q}} - (1-p)\sqrt{\frac{p}{1-p}} - p\sqrt{\frac{1-p}{p}}
\end{align*}
and see that it never becomes lower than 0.
Why did I think this would be more easy.
Now we have to differentiate this mofo.
Anyway I expect this function to have a minimum on the line q=p. There the function is 0. Everywhere else the function is bigger than 0.

\subsection*{(d)}
Take $\ell(0,q)=\sqrt{q}$ and $\ell(1,q)=\sqrt{1-q}$.
\begin{solution}
The associated risk function is given by
\begin{equation*}
L(p,q)=(1-p)\sqrt{q}+p\sqrt{1-q}.
\end{equation*}
Take $p=\tfrac{1}{3}$ and $q=\tfrac{1}{4}$. Then $L(p,p)= \tfrac{1}{9}(2+\sqrt{2})\sqrt{3}\approx 0.657$.
However, we have that $L(p,q)=\tfrac{1}{3}+\tfrac{1}{6}\sqrt{3} \approx 0.622$.
Therefore, $\ell$ is not a proper loss function.
\end{solution}

\section*{Exercise 2}
Let $\phi$ be any differentiable convex function.
\subsection*{(a)}
Show that $B_{\phi}(x,x)=0$.
\begin{solution}
As $\ip{0,x}=0$ for any $x \in  \R^k$, we have that
\begin{align*}
  B_{\phi}(x,x)&=\phi(x)-\phi(x)-\ip{x-x,\nabla \phi(y)} \\
  &=-\ip{0,\nabla \phi(y)} \\
  &=0.
\end{align*}
\end{solution}

\subsection*{(b)}
Show that $B_{\varphi}(x,y)\geq 0$.
\begin{solution}
For fixed $y$, the function $l_y(x)=\phi(y)+\ip{x-y, \nabla \phi(y)}$ denotes the linearization of $\phi$ at the point $y$.
As $\phi$ is convex we have that $\phi(x)\geq l_y(x)$ for all $x,y \in \R^k$. 
It follows that 
\begin{align*}
 B_{\phi}(x,y)&=\phi(x)-\phi(y)-\ip{x-y, \nabla \phi(y)} \\
 &=\phi(x)-l_y(x) \\
 &\geq0.
\end{align*}
\end{solution}

\subsection*{(c)}
Show that $B_{\varphi}(x,y)$ is convex in $x$.
\begin{solution}
\TODO 
\end{solution}

\section*{Exercise 3}
Consider a differentiable convex function $\phi: \triangle_k\rightarrow \mathbb{R}$. Let $\delta^i$ be the $i^{\text{th}}$ standard basis vector (i.e. $\delta^i_i = 1$ and $\delta^i_j = 0$ for $j\neq i$), Define the loss function $\ell_{\phi}:[k]\times \triangle_k \rightarrow \mathbb{R}$ by:
\begin{equation*}
  \ell_{\phi}(i,q) = B_{\phi}\left(\delta^{i},q\right)
\end{equation*}
Show that $\ell$ is proper.
\begin{solution}
We can see that the risk now becomes:
\begin{equation*}
  L(p,q) = \sum^{k}_{i=1}p_iB_{\phi} \left(\delta^{i},q\right)
\end{equation*}
From the definition we know that:
\begin{equation*}
  L(p,q) = \sum^{k}_{i=1}p_i\left(\phi(\delta^i)-\phi(q)-\ip{\delta^i-q,\nabla \phi(q)} \right)
\end{equation*}
Because this function is linear, we can see that

\begin{equation*}
  L(p,q) = \left(\sum^{k}_{i=1}p_i\phi(\delta^i)\right)-\left(\sum^{k}_{i=1}p_i\phi(q)\right)-\left(\ip{\sum^{k}_{i=1}p_i\left(\delta^i-q\right),\nabla \phi(q)}\right)
\end{equation*}
And because $\sum^{k}_{i=1}p_i\delta^i = p$ and $\sum^{k}_{i=1}p_i = 1$ we get that:
\begin{equation*}
  L(p,q)= \left(\sum^{k}_{i=1}p_i\phi(\delta^i)\right) - \phi(q) - \ip{p-q,\nabla \phi(q)}
\end{equation*}
We can also see that:
\begin{align*}
  L(p,p) &=  \left(\sum^{k}_{i=1}p_i\phi(\delta^i)\right) - \phi(p) - \ip{p-p,\nabla \phi(p)}\\
  &=  \left(\sum^{k}_{i=1}p_i\phi(\delta^i)\right) - \phi(p)
\end{align*}
We know that:
\begin{equation*}
  L(p,p)\leq L(p,q) \implies 0\leq L(p,q) - L(p,p)
\end{equation*}
and
\begin{equation*}
  L(p,q) - L(p,p) = \phi(p)-\phi(q)-\ip{p-q,\nabla \phi(q)} = B_{\phi}(p,q) \geq 0
\end{equation*}
Here we recognize the \textit{Bregman Divergence} and we know from exercise 2b that $B_{\phi}(p,q) \geq 0$.


\end{solution}

\section*{Exercise 4}
Ok, ik heb een artikel \citep{frigyik2008functional} toegevoegd waarin dit bewezen wordt. Ik zal het uitpluizen en hier een bewijs geven.



\section*{Exercise 5}
A final grade between $1$ and $10$ is composed of $40\%$ homework
and $60\%$ exams. There are $13$ homework sets of $5$ questions each. For
each homework set, one question is selected uniformly at random and is
graded. There are $2$ exams with $5$ questions each, which are all graded.
Every graded question receives a binary grade: full points or no points.
Imagine a student that hands in all homeworks and exams, and that answers 
each question correctly independently at random with probability $p = 0.8$.
\subsection*{(a)}
What is the probability that the student gets at least grade $7.5$.
Calculate $3$ significant digits.
\begin{solution}
Let $X_1,X_2,\ldots$ be i.i.d. Bernoulli random variables with $\P(X_1=1)=\tfrac{4}{5}$.
Let $B_{13}=\sum_{n=1}^{13}X_n$ and $B_{10}=\sum_{n=14}^{23} X_n$.
Then $B_{13}$ is distributed as a Binomial distribution with parameters $n=13$ and $p=\tfrac{4}{5}$,
and $B_{10}$ is distributed as a Binomial distribution with parameters $n=10$ and $p=\tfrac{4}{5}$.
Note that $B_{13}$ and $B_{10}$ are independent.
We can ommit any non graded homework questions, so the final grade is distributed as 
\begin{equation*}
F=\max\cbrac*{\frac{4}{13}B_{13}+\frac{3}{5} B_{10},1}. 
\end{equation*}
Thus we have that 
\begin{equation}\label{eq:prob_a}
\begin{aligned}
  \P(F \geq 7.5)&=\P(B_{10} = 10)\P(B_{13} \geq 5)+\P(B_{10} = 9)\P(B_{13} \geq 7) \\
 & \hspace{1 em} +\P(B_{10} = 8)\P(B_{13} \geq 9)+\P(B_{10} = 7)\P(B_{13} \geq 11) \\
 & \hspace{1 em} +\P(B_{10} = 6)\P(B_{13} \geq 13).
 \end{aligned}
\end{equation}
We have that 
\begin{equation*}
\P(B_{10} = k)=\binom{10}{k}\paren*{\tfrac{4}{5}}^{k}\paren*{\tfrac{1}{5}}^{10-k} 
\end{equation*}
and that 
\begin{equation*}
\P(B_{13} \geq k)=\sum_{i=k}^{13}\binom{13}{i}\paren*{\tfrac{4}{5}}^{i}\paren*{\tfrac{1}{5}}^{13-i}. 
\end{equation*}
Substituting these expressions in \cref{eq:prob_a} allows for the numerical evaluation of $\P(F \geq 7.5)$.
This yields that $\P(F \geq 7.5)\approx 0.752$.
\end{solution}

\subsection*{(b)}
In an ideal world all questions would be graded. Calculate the probability 
that the same student gets at least grade $7.5$. Calculate $3$
significant digits.
\begin{solution}
 

If every homework question is graded, then the final grade is distributed as
\begin{equation*}
F=\max\cbrac*{\frac{4}{65}B_{65}+\frac{3}{5}B_{10},1}, 
\end{equation*}
where $B_{65}\sim B(65, \tfrac{4}{5})$.
Thus we have that 
\begin{equation}\label{eq:prob_b}
\begin{aligned}
  \P(F \geq 7.5)&=\P(B_{10} = 10)\P(B_{65} \geq 25)+\P(B_{10} = 9)\P(B_{65} \geq 35) \\
 & \hspace{1 em} +\P(B_{10} = 8)\P(B_{65} \geq 44)+\P(B_{10} = 7)\P(B_{65} \geq 54) \\
 & \hspace{1 em} +\P(B_{10} = 6)\P(B_{65} \geq 64).
 \end{aligned}
\end{equation}
Evaluating this expression gives $\P(F \geq 7.5)=0.742$.
\end{solution}


\section*{Appendix}
This is the R-code used in exercise 5(a).
\begin{verbatim}
# Exercise 5 (a)
P <- 0
for(i in 6:10){
  h <- 0
  for(j in (25-2*i):13){
    h <- h + choose(13, j) * 0.8^j * 0.2^(13-j)
  }
  P <- P + (choose(10, i) * 0.8^i * 0.2^(10-i)) * h
}
cat(P)

# Exercise 5 (b)
P <- 0
for(i in 6:10){
  h <- 0
  for(j in (124-10*i):65){
    h <- h + choose(65, j) * 0.8^j * 0.2^(65-j)
  }
  if(i == 10){
    h <- h - choose(65, 24) * 0.8^24 * 0.2^(65-24)
  }
  P <- P + (choose(10, i) * 0.8^i * 0.2^(10-i)) * h
}
cat(P)
\end{verbatim}

\bibliographystyle{plainnat}
\bibliography{MLT} 

\end{document}


